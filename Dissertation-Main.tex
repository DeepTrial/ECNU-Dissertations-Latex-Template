%%%%%%%%%%%%%%%%%%%%%%%%%%%%%%%%%%%%%%%%%%%%%%%%%%%%%%%%%%%%%%%%%%%%%%%%%%%%%
%                                                                           %
%          LaTeX File for Doctor (Master) Thesis of ECNU                    %
%            华东师范大学博士(硕士)论文模板 ____lizb                        %
%                                                                           %
%%%%%%%%%%%%%%%%%%%%%%%%%%%%%%%%%%%%%%%%%%%%%%%%%%%%%%%%%%%%%%%%%%%%%%%%%%%%%
%!TEX program = xelatex
\documentclass[12pt,openany,a4paper,fancyhdr,oneside]{ctexbook}
%\documentclass[12pt,openright,a4paper,fancyhdr,twoside]{ctexbook}
%draft 选项可以使插入的图形只显示外框,以加快预览速度。
%\documentclass[11pt,a4paper,openany,draft]{book}

\usepackage{multirow}
\usepackage{listings}
\usepackage{xcolor}

\usepackage[CJKbookmarks,linkcolor=black,citecolor=black]{hyperref}
\usepackage{pdfpages}
\usepackage{shortvrb,indentfirst,ulem,makeidx}
\usepackage{fancyhdr}
\usepackage{graphicx}
\usepackage{indentfirst,latexsym,amsthm,colortbl,subfigure,clrscode}
\usepackage{algorithm}
% \usepackage{algorithmicx}
% \usepackage{algpseudocode}   % 如果后面出问题,注释掉这个
\usepackage{algorithmic}    % 原本是有这个,和上面的冲突,注释了
\usepackage{bm}                     % 处理数学公式中的黑斜体的宏包
\usepackage{amsmath}                % AMSLaTeX宏包 用来排出更加漂亮的公式
\usepackage{amssymb}                % AMSLaTeX宏包 用来排出更加漂亮的公式

\usepackage{mathrsfs}
\usepackage[subnum]{cases}
\usepackage[numbers,sort&compress]{natbib}
%\usepackage[super,square,numbers,sort&compress]{natbib}
\usepackage{hypernat}
\usepackage{enumerate}
\usepackage{geometry}

\usepackage{times}
\usepackage{fontspec}
\usepackage{libertine}

\usepackage{caption}
\usepackage{titletoc}
%下面是我自己加的包
\usepackage{multirow}
\usepackage{bigdelim}
\usepackage{caption}
\usepackage{amsmath}
\usepackage{graphicx,color,xcolor,colortbl}
\usepackage{epstopdf,epsfig}
\usepackage{amssymb}
\usepackage{tabularx}
\DeclareMathOperator*{\argmax}{argmax}
\DeclareMathOperator*{\argmin}{argmin}
% \usepackage{color}
% \usepackage[dvipsnames]{xcolor}
\hyphenation{op-tical net-works semi-conduc-tor}


\makeindex
\pagestyle{fancy}

\fancyhead[RO,LE]{\wuhao{华东师范大学研究生硕士学位论文}}
\fancyhead[LO]{\wuhao \leftmark} 
\fancyhead[RE]{\wuhao \leftmark}

\renewcommand{\headrulewidth}{0.4pt}
\fancyfoot[CO,CE]{\thepage}

\renewcommand{\algorithmicrequire}{\textbf{Input:}}
\renewcommand{\algorithmicensure}{\textbf{Output:}}
\renewcommand{\algorithmiccomment}[1]{// #1}


%                    根据自己正文需要做的一些定义                 %
%==================================================================
\def\diag{{\rm diag}}
\def\rank{{\rm rank}}
\def\RR{{\cal R}}
\def\NN{{\cal N}}
\def\R{{\mathbb R}}
\def\C{{\mathbb C}}
\let\dis=\displaystyle

\def\p{\partial}
\def\f{\frac}
\def\mr{\mathrm}
\def\mb{\mathbf}
\def\mc{\mathcal}
\def\b{\begin}
\def\e{\end}

\newtheorem{thm1}{Theorem}[part]
\newtheorem{thm2}{Theorem}[section]
\newtheorem{thm3}{Theorem}[subsection]
\newtheorem{them}[thm2]{定理}
\newtheorem{theorem}[thm2]{定理}
\newtheorem{defn}[thm2]{定义}
\newtheorem{define}[thm2]{定义}
\newtheorem{ex}[thm2]{例}
\newtheorem{exs}[thm2]{例}
\newtheorem{example}[thm2]{例}
\newtheorem{prop}[thm2]{命题}
\newtheorem{lemma}[thm2]{引理}
\newtheorem{cor}[thm2]{推论}
\newtheorem{remark}[thm2]{注释}
\newtheorem{notation}[thm2]{记号}
\newtheorem{abbre}[thm2]{缩写}
% \newtheorem{algorithm}[thm2]{算法}
\newtheorem{problem}[thm2]{问题}
\newcommand{\gameauth}{\mathsf{Game}^\mathsf{Auth}}
\newcommand{\gametul}{\mathsf{Game}^\mathsf{TUL}}
\newcommand{\gamepriv}{\mathsf{Game}^\mathsf{PRIV}}
\newcommand{\advaauth}{\mathsf{Adv}_{\mc A}^{\mathsf{Auth}}}
\newcommand{\advatul}{\mathsf{Adv}_{\mc A}^{\mathsf{TUL}}}
\newcommand{\advapriv}{\mathsf{Adv}_{\mc A}^{\mathsf{PRIV}}}
\newcommand{\tsf}{\textsf}
\newtheorem{analysis}{Analysis}
\newcommand{\be}{\begin{enumerate}}
\newcommand{\ed}{\end{enumerate}}

\newcommand{\tabincell}[2]{\begin{tabular}{@{}#1@{}}#2\end{tabular}}
\newcommand{\mbb}{\mathbb}
\newcommand{\yihao}{\fontsize{26pt}{36pt}\selectfont}           % 一号, 1.4 倍行距
\newcommand{\erhao}{\fontsize{22pt}{28pt}\selectfont}          % 二号, 1.25倍行距
\newcommand{\xiaoer}{\fontsize{18pt}{18pt}\selectfont}          % 小二, 单倍行距
\newcommand{\sanhao}{\fontsize{16pt}{24pt}\selectfont}        % 三号, 1.5倍行距
\newcommand{\xiaosan}{\fontsize{15pt}{22pt}\selectfont}        % 小三, 1.5倍行距
\newcommand{\sihao}{\fontsize{14pt}{21pt}\selectfont}            % 四号, 1.5 倍行距
\newcommand{\banxiaosi}{\fontsize{13pt}{19.5pt}\selectfont}    % 半小四, 1.5倍行距
\newcommand{\xiaosi}{\fontsize{12pt}{18pt}\selectfont}            % 小四, 1.5倍行距
\newcommand{\dawuhao}{\fontsize{11pt}{11pt}\selectfont}       % 大五号, 单倍行距
\newcommand{\wuhao}{\fontsize{10.5pt}{15.75pt}\selectfont}    % 五号, 单倍行距



\lstset
{
	basicstyle=\ttfamily,
	% numbers=left,
	% numberstyle=\tiny,
	keywordstyle=\color[RGB]{0, 0, 255},
	commentstyle=\color[RGB]{0, 128, 0},
	frame=shadowbox,
	rulesepcolor=\color{red!20!green!20!blue!20},
	showspaces=false,
	showstringspaces=false,
	extendedchars=false,
	showtabs=false,
	tabsize=4,
	xleftmargin=0.5em,
	xrightmargin=0.5em,
	% aboveskip=1em,
	escapeinside=``
}


%============================ 可以自定义文字块 ================================%

\newcommand{\aaa}{这是我给你们的一个示例}
\newcommand{\bbb}{\aaa \aaa \aaa}
\newcommand{\ccc}{\bbb \bbb \bbb \bbb \bbb

\bbb \bbb \bbb \bbb \bbb }
\newcommand{\abc}{abcdefg1234567890}
\newcommand{\upabc}{ABCDEFGHIJK}
%%% ----------------------------------------------------------------------



%============================= 版芯控制 ================================%
\setlength{\oddsidemargin}{0.57cm}
\setlength{\evensidemargin}{\oddsidemargin}
\voffset-6mm \textwidth=150mm \textheight=230mm \headwidth=150mm
%\rightmargin=35mm
%                                                                       %


%============================= 页面设置 ================================%
%-------------------- 定义页眉和页脚 使用fancyhdr 宏包 -----------------%
% 定义页眉与正文间双隔线
\newcommand{\makeheadrule}{%
\makebox[0pt][l]{\rule[.7\baselineskip]{\headwidth}{0.4pt}}%
\rule[0.85\baselineskip]{\headwidth}{0.4pt} \vskip-.8\baselineskip}
\makeatletter
\renewcommand{\headrule}{%
{\if@fancyplain\let\headrulewidth\plainheadrulewidth\fi
\makeheadrule}} \makeatother

\newcommand{\adots}{\mathinner{\mkern 2mu%
\raisebox{0.1em}{.}\mkern 2mu\raisebox{0.4em}{.}%
\mkern2mu\raisebox{0.7em}{.}\mkern 1mu}}

%\setmainfont{Times New Roman}
\dottedcontents{chapter}[1.5cm]{\xiaosi\heiti}{3.8em}{9.5pt}
\dottedcontents{section}[1.5cm]{\xiaosi\heiti}{2.8em}{9.5pt}


%=============================== 正文部分 ================================%
\begin{document}

\pagestyle{empty}
\setlength{\baselineskip}{25pt}  %%正文设为25磅行间距
\vspace{-2.0cm}
\noindent{{\zihao{4} {\large 2020} 届研究生硕士学位论文}}\\
\vspace{-0.8cm}
\begin{flushleft}
\hspace{-0.5cm}
\renewcommand\arraystretch{1.5}
\begin{tabular}{l}
\noindent{{\zihao{4} 分类号:\underline{\qquad\qquad\qquad\qquad\qquad\qquad}}}  \\
\noindent{{\zihao{4} 密~~~~级:\underline{\qquad\qquad\qquad\qquad\qquad\qquad}}}\\
\end{tabular}
\hskip 1.1cm
\renewcommand\arraystretch{1.5}
\begin{tabular}{l}
\noindent{{\zihao{4} 学校代码:\underline{10269~~~\qquad}}}\\
% \noindent{{\zihao{4} 学~~~~~~~~号:\underline{}}}\\
\noindent{{\zihao{4} 学~~~~~~~~号:\underline{51170000000}}}\\
\end{tabular}
\end{flushleft}


\vskip 1.0cm

\begin{center}
	\hskip 0.5cm
	\scalebox{1.0}{\includegraphics[width=2.7cm]{fig/ecnulogo.png}}
	\scalebox{1.0}{\includegraphics[width=11.5cm,height=2.7cm]{fig/ecnulabel.png}}
	\vskip 0.5cm
	{\textbf{{\xiaoer East China Normal University}}}\\ \vskip 0.2cm
	{\textbf{\erhao 硕~士~学~位~论~文}}\\ \vskip 0.2cm
	{\textbf{{\xiaoer MASTER'S DISSERTATION}}}\\\end{center}





\vskip 1.0cm

%基于分数阶微分和特征的一致性图像配准研究
\begin{center}
{\erhao \bf 论文题目:\underline{~~你的论文标题在这里~~}}\\
{\erhao \bf ~~~~~~~~~~~~~~~~~~~~\underline{~~你的论文标题在这里~~}}
%{\erhao \bf ~~~~~~~~~~~~~~~~~~~~\underline{~~~~~~~~~~~~~~~~}}
\end{center}
\vskip 1.0cm
\begin{center}

\renewcommand\arraystretch{1.5}
	\begin{tabular}{l}
{\sihao \bf 院\qquad\ \ \ 系:}\\
{\sihao \bf 专~业~名~称:}\\
{\sihao \bf 研~究~方~向:}\\
{\sihao \bf 指~导~教~师:}\\
{\sihao \bf 学位申请人:}
\end{tabular}
\begin{tabular}c
{\sihao \bf  ~~计算机科学与技术学院}               \\
\hline {\sihao \bf ~~计算机科学与技术}              \\
\hline {\sihao \bf ~~图像处理与模式识别~~}\\
% \hline {\sihao \bf ~~ \ \ }  \\
% \hline{\sihao \bf  ~~}      \\
\hline {\sihao \bf ~~XX~~教授\ \ }  \\
\hline{\sihao \bf  ~~XX}      \\
\hline
\end{tabular}


\end{center}

\vskip 1.95cm
\begin{center}
{\sihao 2020年X月X日}
\end{center}

%\clearpage\ \newpage
\newpage

\pagestyle{empty}

%\noindent{\large Dissertation for Master Degree in 2020}
%\hskip 1cm {\large University Code: 10269}\\
%\hspace *{\fill}{\large Student Number: 51174506034}
\noindent{\large 2020 MASTER’S  DISSERTATION}
\hskip 1.4cm {\large School Code: 10269}\\
\hspace*{\fill}{\large Student Number: 51170000000}

\vskip 2cm

\begin{center}
{\Huge $\mathbb{EAST}\,\mathbb{CHINA}\,\mathbb{NORMAL}\,
\mathbb{UNIVERSITY}$}
\end{center}

\vskip 3cm

%\begin{center}
%{\erhao\bf  Title: \underline{The Combination of Side Window Fractional}}\\
%{\erhao\bf  \underline{~~~~Order Derivate and Feature with Inertial~~~}}\\
%{\erhao\bf  \underline{Constraint for Non-rigid Image Registration}}
%\end{center}
\begin{center}
{\erhao\bf  Title: \underline{Research on XXXXX XXXXXX }}\\
{\erhao\bf  \underline{~~~XXXXXXX XXXXXXX XXXXX~~~}}\\
{\erhao\bf  \underline{~~~~~~XXXX XXXXXXXX XXXXXXX~~~~~~~}}
\end{center}

\vskip 2cm {\large
\begin{center}
\begin{tabular}{l}
Department:\\
Major:\\
Research Area:\\
Supervisor:\\
Candidate:
\end{tabular}
\begin{tabular}c
~~~School of Computer Science and Technology \\
\hline ~~~Computer Science and Technology  \\
\hline ~~~Image Processing and Pattern Recognition\\
% \hline ~~~  \\
% \hline ~~~ \\
\hline ~~~XXXX~ Professor  \\
\hline ~~~XXXX  \\
\hline
\end{tabular}
\end{center}}

\vskip 30mm

\begin{center}
{\Large Month, 2020}
\end{center}

%\clearpage\ \newpage
\input{A3-COPYRIGHT.tex}
%\clearpage\ \newpage
\newpage
\pagestyle{empty}
$$\\ \\ \\ $$

\centerline{\bf\Large $\underline{\mbox{\kaishu {}}}\,\,
	$硕士学位论文答辩委员会成员名单}

\vskip 10mm

\begin{center}
	{\large
		\begin{tabular}{| p{25mm}| p{30mm}| p{48mm}| p{25mm}|}\hline
			\vfill\hfill{\heiti 姓名}\hspace*{\fill} &\vfill\hfill{\heiti 职称}\hspace*{\fill} &
			\vfill\hfill{\heiti 单位}\hspace*{\fill} &\vfill\hfill{\heiti 备注}\hspace*{\fill}\\[6pt]\hline
			\vfill\hfill{\kaishu XXXX}\hspace*{\fill} &\vfill\hfill{\kaishu 教授}\hspace*{\fill} &\vfill\hfill{\kaishu 华东师范大学}\hspace*{\fill} & \vfill\hfill {\kaishu 主席}\hspace*{\fill} \\[6pt]\hline
			\vfill\hfill{\kaishu XXXX}\hspace*{\fill} &\vfill\hfill{\kaishu 教授}\hspace*{\fill} &\vfill\hfill{\kaishu 华东师范大学}\hspace*{\fill} & \vfill{\heiti }\\[6pt]\hline
			\vfill\hfill{\kaishu XXXX}\hspace*{\fill} &\vfill\hfill{\kaishu 副教授}\hspace*{\fill} &\vfill\hfill{\kaishu 华东师范大学}\hspace*{\fill} & \vfill{\heiti }\\[6pt]\hline
			\vfill\hfill{\kaishu}\hspace*{\fill} &\vfill\hfill{\kaishu }\hspace*{\fill} &\vfill\hfill{\kaishu}\hspace*{\fill} & \vfill{\heiti }\\[6pt]\hline
			\vfill\hfill{\kaishu}\hspace*{\fill} &\vfill\hfill{\kaishu}\hspace*{\fill} &\vfill\hfill{\kaishu}\hspace*{\fill} & \vfill{\heiti }\\[6pt]\hline
			\vfill\hfill{\kaishu}\hspace*{\fill} &\vfill\hfill{\kaishu}\hspace*{\fill} &\vfill\hfill{\kaishu}\hspace*{\fill} & \vfill{\heiti }\\[6pt]\hline
		\end{tabular}
	}
\end{center}


%\clearpage\ \newpage

%\newpage
\pagenumbering{Roman}  %摘要页的页码是罗马字符 roman是小写  Roman是大写
\pagestyle{plain}

\addcontentsline{toc}{chapter}{摘要}   %在目录页添加摘要及页码,如不需要可直接注释
\vspace{-2.5cm}
\chapter*{\zihao{2}\heiti{摘~~~~要}}
%\vskip 1cm
%\vspace{-1cm}

论文摘要论文摘要论文摘要论文摘要论文摘要论文摘要论文摘要论文摘要论文摘要论文摘要论文摘要论文摘要论文摘要论文摘要论文摘要论文摘要论文摘要论文摘要论文摘要论文摘要论文摘要论文摘要论文摘要论文摘要论文摘要论文摘要论文摘要论文摘要论文摘要论文摘要论文摘要论文摘要论文摘要论文摘要论文摘要论文摘要论文摘要论文摘要论文摘要论文摘要论文摘要论文摘要论文摘要论文摘要论文摘要论文摘要论文摘要论文摘要论文摘要论文摘要论文摘要论文摘要论文摘要论文摘要论文摘要论文摘要论文摘要论文摘要论文摘要论文摘要论文摘要论文摘要

本文的主要工作包括:
\begin{enumerate}
	\item
	\textbf{创新点1创新点1创新点1}论文摘要论文摘要论文摘要论文摘要论文摘要论文摘要论文摘要论文摘要论文摘要论文摘要论文摘要论文摘要论文摘要论文摘要论文摘要论文摘要论文摘要论文摘要论文摘要论文摘要论文摘要论文摘要论文摘要论文摘要论文摘要论文摘要论文摘要论文摘要论文摘要论文摘要论文摘要
	
	\item
	\textbf{创新点2创新点2创新点2}论文摘要论文摘要论文摘要论文摘要论文摘要论文摘要论文摘要论文摘要论文摘要论文摘要论文摘要论文摘要论文摘要论文摘要论文摘要论文摘要论文摘要论文摘要论文摘要论文摘要论文摘要论文摘要论文摘要论文摘要论文摘要论文摘要论文摘要论文摘要论文摘要论文摘要论文摘要
		
	\item
	\textbf{创新点3创新点3创新点3}论文摘要论文摘要论文摘要论文摘要论文摘要论文摘要论文摘要论文摘要论文摘要论文摘要论文摘要论文摘要论文摘要论文摘要论文摘要论文摘要论文摘要论文摘要论文摘要论文摘要论文摘要论文摘要论文摘要论文摘要论文摘要论文摘要论文摘要论文摘要论文摘要论文摘要论文摘要
\end{enumerate}

论文摘要论文摘要论文摘要论文摘要论文摘要论文摘要论文摘要论文摘要论文摘要论文摘要论文摘要论文摘要论文摘要论文摘要论文摘要论文摘要论文摘要论文摘要论文摘要论文摘要论文摘要论文摘要论文摘要论文摘要论文摘要论文摘要论文摘要论文摘要论文摘要论文摘要论文摘要

\hspace{-0.5cm}
\sihao{\heiti{关键词:}} \xiaosi{关键词,keyword,关键词,keyword,关键词,keyword}


\addcontentsline{toc}{chapter}{ABSTRACT}  %在目录页添加abstract及页码,如不需要可直接注释
\newpage
\vspace{-1cm}
\chapter*{\zihao{-2}\heiti{ABSTRACT}}
%\vspace{-0.5cm}
Your abstract. Your abstract. Your abstract. Your abstract. Your abstract. Your abstract. Your abstract. Your abstract. Your abstract. Your abstract. Your abstract. Your abstract. Your abstract. Your abstract. Your abstract. Your abstract. Your abstract. Your abstract. Your abstract. Your abstract. Your abstract. Your abstract. Your abstract. Your abstract. Your abstract. Your abstract.

The main work of this paper includes:
\begin{enumerate}
\item 
\textbf{First of your idea.First of your idea.First of your idea.First of your idea.First of your idea.}
Your abstract. Your abstract. Your abstract. Your abstract. Your abstract. Your abstract. Your abstract. Your abstract. Your abstract. Your abstract. Your abstract. Your abstract. Your abstract. Your abstract. Your abstract. Your abstract. Your abstract. Your abstract. Your abstract. Your abstract. Your abstract. Your abstract. Your abstract. Your abstract. Your abstract. Your abstract.
	
\item
\textbf{Second of your idea.Second of your idea.Second of your idea.Second of your idea.Second of your idea.} Your abstract. Your abstract. Your abstract. Your abstract. Your abstract. Your abstract. Your abstract. Your abstract. Your abstract. Your abstract. Your abstract. Your abstract. Your abstract. Your abstract. Your abstract. Your abstract. Your abstract. Your abstract. Your abstract. Your abstract. Your abstract. Your abstract. Your abstract. Your abstract. Your abstract. Your abstract.

\item
\textbf{Third of your idea. Third of your idea.Third of your idea. Third of your idea. Third of your idea.} Your abstract. Your abstract. Your abstract. Your abstract. Your abstract. Your abstract. Your abstract. Your abstract. Your abstract. Your abstract. Your abstract. Your abstract. Your abstract. Your abstract. Your abstract. Your abstract. Your abstract. Your abstract. Your abstract. Your abstract. Your abstract. Your abstract. Your abstract. Your abstract. Your abstract. Your abstract.
\end{enumerate}

Your abstract. Your abstract. Your abstract. Your abstract. Your abstract. Your abstract. Your abstract. Your abstract. Your abstract. Your abstract. Your abstract. Your abstract. Your abstract. Your abstract. Your abstract. Your abstract. Your abstract. Your abstract. Your abstract. Your abstract. Your abstract. Your abstract. Your abstract. Your abstract. Your abstract. Your abstract.

%空行不能省
%\hspace{-0.5cm}
{\sihao{\textbf{Keywords:}}} \textit{keywords; keywords; keywords; keywords}




































\setcounter{tocdepth}{2}

\addcontentsline{toc}{chapter}{目录}  %在目录页添加目录及页码,如不需要可直接注释
\tableofcontents

\addcontentsline{toc}{chapter}{插图}  %在目录页添加插图及页码,如不需要可直接注释
\listoffigures

\addcontentsline{toc}{chapter}{表格}  %在目录页添加表格及页码,如不需要可直接注释
\listoftables

\newpage
\pagenumbering{arabic}
\pagestyle{fancy}


\CTEXsetup[format+={\zihao{3}\heiti}]{chapter}
\CTEXsetup[format+={\raggedright\zihao{4}\heiti}]{section}
\CTEXsetup[format+={\zihao{-4}\heiti}]{subsection}


\setlength{\baselineskip}{25pt}  %%正文设为25磅行间距

\chapter{绪论}
\label{ch1}
\section{研究背景与意义}
研究背景与意义研究背景与意义研究背景与意义研究背景与意义研究背景与意义研究背景与意义研究背景与意义研究背景与意义研究背景与意义研究背景与意义研究背景与意义研究背景与意义研究背景与意义研究背景与意义研究背景与意义研究背景与意义研究背景与意义研究背景与意义\cite{ReferenceName}。如图\ref{fig:traditional_framework}所示。

\begin{figure}[H]\vspace{0pt}
	\centering
	\includegraphics[width=0.95\linewidth]{fig/timg}
	\caption{图片的题注}
	\label{fig:traditional_framework}\vspace{0pt}
\end{figure}

\subsection{小标题1}
小标题1小标题1小标题1小标题1小标题1小标题1小标题1小标题1小标题1小标题1小标题1小标题1小标题1小标题1小标题1小标题1小标题1小标题1小标题1小标题1小标题1小标题1小标题1小标题1小标题1小标题1小标题1小标题1小标题1小标题1小标题1小标题1小标题1小标题1小标题1小标题1小标题1小标题1小标题1小标题1小标题1

\subsection{小标题2}
小标题2小标题2小标题2小标题2小标题2小标题2小标题2小标题2小标题2小标题2小标题2小标题2小标题2小标题2小标题2小标题2小标题2小标题2小标题2小标题2小标题2小标题2小标题2小标题2小标题2小标题2小标题2小标题2小标题2小标题2小标题2小标题2小标题2小标题2小标题2小标题2小标题2小标题2小标题2小标题2小标题2小标题2小标题2小标题2小标题2小标题2小标题2小标题2小标题2小标题2小标题2小标题2小标题2小标题2小标题2小标题2小标题2小标题2小标题2小标题2。如表\ref{tab:table_eg_1}所示

\begin{table}[h!]
	\centering
	\caption{表格的题注}
	\label{tab:table_eg_1}
	\begin{tabular}{l|ll}
		\hline
		xx & xx & xx \\ \hline
		xx & 11 & 11 \\
		xx & 11 & 11 \\ \hline
	\end{tabular}
\end{table}

\section{本文主要工作}
本文主要工作本文主要工作本文主要工作本文主要工作本文主要工作本文主要工作本文主要工作本文主要工作本文主要工作本文主要工作本文主要工作本文主要工作本文主要工作本文主要工作本文主要工作本文主要工作本文主要工作本文主要工作本文主要工作本文主要工作本文主要工作本文主要工作本文主要工作本文主要工作


%\input{C1-CHAP2.tex}
%\input{C1-CHAP3.tex}
%\input{C1-CHAP4.tex}
%\input{C1-CHAP5.tex}
%\input{C1-CHAP6.tex}
%\input{C8-CHAP8.tex}
%\appendix

\chapter{附录}
\vspace{-1cm}







\pagestyle{plain}
\clearpage
\phantomsection
\addcontentsline{toc}{chapter}{参考文献}
% \begin{thebibliography}{zz}




% \bibitem{2000Joux}
% Joux, Antoine. "A one round protocol for tripartite Diffie–Hellman."Algorithmic number theory. Springer Berlin Heidelberg, 2000. 385-393.

% \bibitem{2001Boneh}

% Boneh, Dan, and Matt Franklin. "Identity-based encryption from the Weil pairing." Advances in Cryptology—CRYPTO 2001. Springer Berlin Heidelberg, 2001.

% \bibitem{1984S}
% Identity-Based Cryptosystems and Signature Schemes

% \bibitem{2003J}
% J.C.Cha,J.H.Cheon.An identity-based signature from gap Diffie-Hellman groups.In Y Desmedt,editor,Public Key Cryptography-PKC 2003,LNCS 2567,Berlin:Springer-Verlag,2003:18-30



% \end{thebibliography}


\bibliographystyle{bib/GBT7714-2005}
\bibliography{bib/tex}

%\pagestyle{plain}\clearpage
\pagestyle{plain}
\clearpage
\phantomsection
\addcontentsline{toc}{chapter}{致谢}
{\fangsong
	\chapter*{致\qquad 谢}\vskip 2mm
	\vspace{-1cm}
	\large{致谢致谢致谢致谢致谢致谢致谢致谢致谢致谢致谢致谢致谢致谢致谢致谢致谢致谢致谢致谢致谢致谢致谢致谢致谢致谢致谢致谢致谢致谢致谢致谢致谢致谢致谢致谢致谢致谢致谢致谢致谢致谢致谢致谢致谢致谢致谢致谢



%\vspace{0.2cm} \hspace{9.8cm}  
%徐~~诚
%
%\hspace{9cm}  二零二零年三月
  

	}
	
	\vspace{0.2cm}
	

}

\pagestyle{plain}
\clearpage
\phantomsection
\addcontentsline{toc}{chapter}{发表论文和科研情况}
\chapter*{\large 攻读硕士学位期间参与的项目以及学术成果}
\vskip 2mm
\vspace{-1cm}
\renewcommand{\labelenumi}{[\arabic{enumi}]}


{\heiti $\blacksquare$ 已完成学术论文}\vskip 3mm
\begin{enumerate}
	\item \textbf{XXXXXXX}, XXXXXXXX XXXXXXXXXXXX XXXXXXXXXXXXXXX XXXXXX XXXXXXXXXX XXXXXX
	
%	\item Gabor Feature Based LogDemons with Inertial Constraint for Nonrigid Image Registration. IEEE Transactions on Image Processing. (Major Revision)
\end{enumerate} 




\printindex
\end{document}

